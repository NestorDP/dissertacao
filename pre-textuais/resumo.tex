% ---
% RESUMOS
% ---

% resumo em português
\setlength{\absparsep}{18pt} % ajusta o espaçamento dos parágrafos do resumo
\begin{resumo}
    Os novos projetos em robótica têm exigido cada vez mais poder de processamento, consequentemente, rquerem uma maior eficiência energética, principalmente nas aplicações que fazem uso de baterias. Dessa maneira o uso do FPGA pode contribuir com ganho de poder de processamento associado ao baixo consumo. Neste trabalho foi elaborado um método para estabelecer a comunicação entre o ROS e um FPGA embarcado em um SoC da família Cyclone V. Por meio de um sistema servidor-cliente, através de um link Gigabit ethernet, foi possível estabelecer a comunicação entre os elementos do sistema. Os testes de desempenho foram realizados no kit de desenvolvimento DE10-Nano e alcançaram um resultado considerado aceitável.

    \vspace{\onelineskip}

 \textbf{Palavras-chave}: ROS, SoC, FPGA, Cyclone V.
\end{resumo}
