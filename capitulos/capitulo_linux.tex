\chapter{\textit{Embedded Linux}}

%O que é um sistema operacional de tempo real?
% Porque usar um RTOS?
    %apresentar outras técnicas alternativas ao uso de um RTOS

O FreeRTOS é um sistema de tempo real de grande sucesso desde de seu início,
é suportado por mais de 35 architectures e ``foi baixado uma vez a cada 175
segundos no ano de 2018'' \cite{FreeRTOS_history}. O FreeRTOS é um projeto de
\textit{free} e de código aberto disponível sobe licença MIT. O FreeRTOS foi
a escolha para esse trabalho por não possuir restrições ao seu uso e também
por sua popularidade, o que facilita a busca por informações de seu uso uma
tarefa menos árdua.

\section{Distribuição rsyocto}

FreeRTOS é um kernel de um sistema operacional de tempo real para sistemas
embarcados, permite que aplicações sejam organizadas como coleções de 
independentes threads mantendo os requisitos de tempo real do sistema.
Foi desesenvolvido em 2003, inicialmente por Richard Barry e posteriormente
mantida pela \textit{Real Time Engineers Ltd.}, em 2017 o projeto 
FreeRTOS passou aser adminitrado \textit{Amazon Web Services}.

O kernel básico é formado por 

 

% \section{Gerenciamento de Memória}

% \section{Tasks}

% \section{Interrupções}




% \section{FreeRTOS+TCP}