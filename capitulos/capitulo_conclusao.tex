\chapter{Conclusão}\label{cap:conclusao}

Os resultados alcançados para os dois ensaios foram considerados satisfatórios, apesar de, na transferência das imagens, ter sido observado um pequeno \textit{delay}, menor do que 10 ms. Um atraso desse nível não impossibilitaria, por exemplo, o uso do sistema em um robó móvel que usasse o sistema para realizar, através do FPGA, cálculos de odometria visual ou mesmo planejamento de rotas. 

Portanto,  objetivo de estabelecer uma comunicação com alta taxa de transferência entre o \textit{framework} de robótica ROS e um SoC com FPGA integrado foi alcançado, possibilitando o uso de aceleração por hardware de maneira mais ágil em projetos de robótica.
