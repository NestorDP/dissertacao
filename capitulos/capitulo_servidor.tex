\chapter{Servidor}\label{cap:server}

% Introdução ao programa servidor
%=====================================
No modelo cliente-servidor, o servidor é o elemento responsável por executar uma ação somente após um pedido realizado pelo cliente, sendo assim, o servidor deve estar sempre preparado para responder à uma solicitação de serviço. Deste modo, além da capacidade de se comunicar com o cliente, o servidor deve oferecer algum serviço de interesse do cliente. 

Neste trabalho o programa servidor disponibiliza ao cliente a interface com o FPGA presente no SoC. Assim, ao receber o pedido do cliente, o servidor deve ser capaz de enviar os dados vindos do cliente à unidade de processamento embarcada no FPGA e em seguida devolver esses dados já processados ao cliente. Entretanto, antes mesmo de iniciar o desenvolvimento do código do servidor, devemos compilar uma distribuição Linux e configurar o processo de boot desse sistema no processador ARM presente no SoC. Estas etapas do desenvolvimento do servidor serão descritas neste capítulo.

% explicar processo de boot
%=====================================
\section{Processo de BOOT}
O fluxo de boot do Linux na placa é resumido na Figura~\ref{fig:linux}. O primeiro elemento de software é o \textit{Boot ROM} que está gravado de fábrica internamente no dispositivo. Os arquivos de \textit{PreLoader},\textit{ U-boot} e do sistemas de aquivo do linux são salvos em um cartão de memória micro SD\@. O \textit{Secondary Program Loader - SPL}, conhecido como \textit{PreLoader}, é executado a partir da Boot ROM\@. Ele é responsável por configurar o sistema para que o \textit{bootloader} (U-boot) possa ser executado. A Intel fornece uma ferramenta chamada BSP editor que, a partir de arquivos que descrevem o hardware, pode gerar o PreLoader para o projeto específico~\cite{SocLinux}.  

\begin{figure}[ht]
	\caption{Boot linux embarcado}
	\begin{center}
		\includegraphics[scale=0.65]{imagens/embeddedLinux.png}\\
		{\small \textbf{Fonte:}\cite{SocLinux}}
    \end{center}\label{fig:linux}
\end{figure}

A etapa seguinte ao \textit{PreLoader} é o \textit{bootloader}. Nessa fase do \textit{boot} todas as questões de baixo nível do SoC, como por exemplo, os clocks, pinos e SDRAM, já foram inicializados e estão prontos. O objetivo do bootloader, é obter essas informações do sistema e fazer com que ele funcione até o ponto onde o Linux possa ser iniciado. Outra função importante do U-boot em um SoC Intel é programar o FPGA~\cite{SocLinux}.

% Explicar a distribuição usada
%=====================================
\section{Distribuição Linux rsyocto}
Com todos os arquivos necessários para o boot do sistema já gerados a partir das ferramentas de desenvolvimento da Intel, poderemos escolher nosso sistema operacional. O sistema escolhido foi uma distribuição linux desenvolvida exclusivamente para SoCs da Intel com FPGA integrado, como pode ser observado na descrição da distribuição disponível em~\cite{rsyocto}:

\begin{citacao}
	``\textbf{rsyocto} is an open source Embedded Linux Distribution designed with the Yocto Project and with a custom build flow to be optimized for Intel SoC-FPGAs (Intel Cyclone V and Intel Arria 10 SX SoC-FPGA with an ARM Cortex-A9) to achieve the best customization for the strong requirements of modern embedded SoC-FPGA applications.''
\end{citacao} 

O \textit{rsyocto} faz uso do Kernel Linux~\textbf{linux-socfpga 5.11}~\cite{linuxsocfpga} e possui um conjunto de ferramentas necessárias para ajudar a simplificar o uso e desenvolvimento de aplicações desenvolvidas para os SoCs com FPGA integrado da Intel. Além destas ferramentas o \textit{rsyocto} dispõe de \textit{drivers} para todos os periféricos de comunicação integrados SoC, como por exemplo, a interfaces I2C e CAN, e para todas as interfaces entre o HPS e o FPGA\@. O \textit{rsyocto} oferece um conjunto de simples aplicações que podem ser executadas em linha de comando, que possibilitam executar novas configurações no FPGA, usando o \textit{FPGA Manager}, até mesmo ler e escrever a interface \textit{ARM AXI-Bridge} que permite interagir com o FPGA\@. 

Essas aplicações simplificam a comunicação com o FPGA a partir de simples comandos que podem ser executados a partir do terminal, através de todas as interfaces disponíveis. Estes comandos são:~\textbf{lwhps2fpga}, que permite ações de leitura e escrita no barramento \textit{Lightweight HPS-to-FPGA-Bridge}, o~\textbf{hps2fpga} para leitura e escrita no barramento \textit{HPS-to-FPGA-Bridge}, o~\textbf{hps2sdram}, que proporciona acesso à interface da SDRAM e os~\textbf{gpi} e~\textbf{gpo} para acesso aos sinais de uso geral. Na Figura~\ref{fig:rsyocto} podemos ter uma visão geral de todas as ferramentas disponíveis na distribuição linux~\textit{rsyocto}.


\begin{figure}[ht]
	\caption{Visão geral da distribuição Linux \textit{rsyocto}}
	\begin{center}
		\includegraphics[scale=0.55]{imagens/rsYoctoLayers.jpg}\\
		{\small \textbf{Fonte:}\cite{rsyocto}}
    \end{center}\label{fig:rsyocto}
\end{figure}

A seguir são listadas algumas características da distribuição linux rsyocto que fizeram dela uma excelente escolha para uso neste trabalho:
\begin{itemize}
	\item Linux embarcado especialmente desenvolvido para SoC-FPGAda Intel;
	\item Linux Kernel 5.11;
	\item Acesso total ao Dual-Core ARM (ARMv7-A) Cortex-A9;
	\item Configuração da malha FPGA durante o \textit{boot} e com comandos Linux simples;
	\item Todas as interfaces entre o HPS e o FPGA estão habilitadas e prontas para uso;
	\item Ferramentas para interagir com a malha do FPGA a partir das \textit{ARM AXI HPS-toFPGA bridges};
	\item Componentes Hard IP (I²C-,SPI-, CAN-BUS or UART) do HPS estão roteados para o FPGA\@;
	\item Suporte para o USB \textit{Host} com ferramentas de teste (lsusb);
	\item Interface de rede com suporte ao iPv4 dinâmico e estático;
	\item Compilador gcc 9.3.0; glibc e glib-2.0 da \textit{GNU C Library}, cmake 3.16.5
	\item Python 3.8 Python3-dev e Python-dev;
	\item git 2.31, wget 1.20.3, curl 7.69.1; 
	\item Gerenciador de pacotes opkg configurado para pacotes de diferentes distribuições Linux; e
	\item Checagem e execução do \textit{FPGA-Configuration} da malha do FPGA\@.
	\end{itemize}

% Explicar o códgo do servidor
%=====================================
\section{Servidor: \textit{Interface socket server}}

Com o Linux já devidamente configurado, ele já pode ser inicializado no processador do SoC, assim podemos iniciar o desenvolvimento do servidor. Para manter o independência entre os componentes do sistema, afim de garantir a sua modularidade, o código do servidor não se baseia no ROS\@, ou seja, não é necessário haver o uma instância do ROS rodando no processador do SoC\@. 

Através da biblioteca compartilhada libinterfacesocket, descrita anteriormente, podemos manter o padrão da comunicação de rede entre o servidor e o clientes. Sendo assim, para o desenvolmento do servidor, é preciso, inicialmente, realizar o download do código fonte da biblioteca de comunicação, a libinterfacesocket, o código fonte pode ser baixado em~\cite{interface-socket-server}. CApós o download, devemos realizar a compilação e a instalação da libinterfacesocket para que o servidor tenha acesso às suas funcionalidades. 

\begin{figure}[ht]
	\caption{Fluxograma simplificado do servidor}
	\begin{center}
		\includegraphics[scale=0.41]{imagens/fluxogramaServidor.png}\\
		{\small \textbf{Fonte:}Elaborado pelo autor}
    \end{center}\label{fig:fluxoServidor}
\end{figure} 

Logo que é inicializado o servidor mantém uma porta aberta para estabelecer uma conexão com o cliente e, assim que a conexão é estabelecida, o cliente já pode começar a enviar os dados. Ao receber os dados do cliente, o servidor envia-os ao FPGA que os processa e os devolve ao servidor para que ele possa reenviar os estes dados, já processados, ao cliente. Um fluxograma simplificado desse processo pode ser visualizado na Figura \ref{fig:fluxoServidor}.

O acesso do servidor ao FPGA é feito através de mapeamento de endereços do linux, uma vez que os SoCs da Intel possuem uma arquitetura de memória mapeada. Desta forma, além da biblioteca desenvolvida durante a pesquisa, a libinterfacesocket, um outro arquivo cabeçalho é necessário para facilitar o acesso aos endereços de memória do HPS\@. 

Quando uma instância do processador ARM é incluída no projeto do Platform Designer, Figura~\ref{fig:Platform}, é gerado um arquivo.sopcinfo no momento em que o projeto é compilado. Podemos usar esse arquivo como entrada da ferramenta \textit{sopc-create-header-files}, presente no \textit{SoC FPGA Embedded Development Suite} - SoC EDS, para gerar um novo arquivo com extensão \textit{.h}. Este arquivo cabeçalho lista os endereços base de todos os módulos IP incluídos no FPGA\@. 

\begin{figure}[ht]
	\caption{Platform Designer}
	\begin{center}
		\includegraphics[scale=0.35]{imagens/Platiform.png}\\
		{\small \textbf{Fonte:}Elaborado pelo autor}
    \end{center}\label{fig:Platform}
\end{figure} 

Portanto, o arquivo cabeçalho gerado disponibiliza o \textit{offset} do endereço em que cada periférico está localizado, incluindo os novos môdulos periféricos desenvolvidos, para cada uma das FPGA-bridges disponíveis. Deste modo o servidor pode enviar ou receber dados ao periférico conhecendo o endereço da FPGA-bridge em que este periférico está conectado e o seu nome, sem a necessidade de se conhecer exatamente o seu endereço. Na Figura~\ref{fig:codigodobro} é apresentada uma parte do arquivo cabeçalho gerado pela ferramenta \textbf{sopc-create-reader-files}, neste trecho são realizadas algumas definições que facilitam o acesso ao IP dobro, que já está instanciado no FPGA\@. Este módulo IP foi usado para testar o processo completo de comunicação entre o tópico ROS até o FPGA\@.


\begin{figure}[ht]
\caption{Definição de endereço presente no arquivo hps\underline{ }0.h}
\begin{center}
\begin{lstlisting}[language=C++, backgroundcolor=\color{gray!10}]
 /*
 * Macros for device 'Dobro_0', class 'Dobro'
 * The macros are prefixed with 'DOBRO_0_'.
 * The prefix is the slave descriptor.
 */
 #define DOBRO_0_COMPONENT_TYPE Dobro
 #define DOBRO_0_COMPONENT_NAME Dobro_0
 #define DOBRO_0_BASE 0x38
 #define DOBRO_0_SPAN 4
 #define DOBRO_0_END 0x3b
\end{lstlisting}
{\small \textbf{Fonte:}Elaborado pelo autor}	
\end{center}\label{fig:codigodobro}
\end{figure}


Para interagir com FPGA usando o recurso de memória mapeada, a primeira coisa que devemos fazer é abrir o dispositivo de memória do sistema. Em distribuições linux o acesso à memória física do sistema é realizado através do dispositivo /dev/mem, esse dispositivo funciona como um arquivo de texto que representa uma imagem da memória principal do computador~\cite{manmem}. Por se comportar como um arquivo no sistema, é possível assim, abri-lo como abriríamos qualquer arquivo de texto, com a funções \textit{open}, como pode ser visto na linha 2 da Figura~\ref{fig:codigomap}.

\begin{figure}[ht]
	\caption{Memória mapeada}
	\begin{center}
	\begin{lstlisting}[language=C++, backgroundcolor=\color{gray!10}]
	 // Open up the /dev/mem device
	 devmem_fd = open("/dev/mem", O_RDWR | O_SYNC);
	 if(devmem_fd < 0) {
		perror("devmem open"); 
		exit(EXIT_FAILURE);
	 }	
	 lw_bridge_map = (uint32_t *)mmap( NULL, 
					   HPS_TO_FPGA_AXI_SPAN, 
					   PROT_READ|PROT_WRITE, 
					   MAP_SHARED, 
					   devmem_fd, 
					   HPS_TO_FPGA_AXI_BASE ); 
									
	 if(lw_bridge_map == MAP_FAILED) {
		perror("devmem mmap");
		close(devmem_fd);
		exit(EXIT_FAILURE);
	 }
	\end{lstlisting}
	{\small \textbf{Fonte:}Elaborado pelo autor}	
	\end{center}\label{fig:codigomap}
	\end{figure} 

	
Em seguida podemos usar o descritor de arquivo gerado como retorno da função open, para criar um ponteiro, que dará acesso à memória principal do sistema. Para gerarmos este ponteiro, podemos usar a função \textit{mmap}, a qual cria um mapeamento virtual no espaço de endereço passado como argumento da função. Nos dois primeiros argumentos da função \textit{mmap}, como pode ser visto na linha 10 da Figura~\ref{fig:codigomap},  podemos definir o espaço de endereços para o mapeamento. No código apresentado este espaço se inicial no endereço zero, NULL\@, e termina na constante HPS\underline{ }TO\underline{ }FPGA\underline{ }AXI\underline{ }SPAN (0x3C000000), que guarda o valor do alcance máximo do espaço de endereço das interfaces entre o HPS e o FPGA\@. O último argumento da função \textit{mmap} é o endereço base da \textit{Lightweight HPS-to-FPGA-Bridge} (0xC0000000), representado pela constante HPS\underline{ }TO\underline{ }FPGA\underline{ }AXI\underline{ }BASE\@.

Agora que já existe um ponteiro para a região de memória onde se encontra a interface de comunicação com o FPGA, podemos enviar e receber dados para o IP instanciado no FPGA\@. Para isto, vamos usar este ponteiro para o endereço da \textit{HPS-to-FPGA-Bridge}, somado com o \textit{offset} presente no arquivo hps\underline{ }0.h.  Consequentemente, será possível interagir com o IP em um programa escrito em linguagem C, da mesma maneira que interagimos com um ponteiro comum. O bloco de código da Figura~\ref{fig:codigomapdobro} ilustra este processo de escrita e leitura de dados em um IP através de um ponteiro.

\begin{figure}[ht]
\caption{Memória mapeada}
\begin{center}
\begin{lstlisting}[language=C++, backgroundcolor=\color{gray!10}]
 // Set the dobro_map to the correct offset within the RAM
 uint32_t * dobro_map = 0;
 dobro_map = (uint32_t*)(lw_bridge_map + DOBRO_0_BASE);

 *dobro_map = 56;
 valor = *dobro_map;
\end{lstlisting}
{\small \textbf{Fonte:}Elaborado pelo autor}	
\end{center}\label{fig:codigomapdobro}
	
\end{figure}
