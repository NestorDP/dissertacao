
\chapter{Teoria}

\section{Quadros}



% \begin{quadro}[htb]
% \caption{\label{quadro_exemplo}Exemplo de quadro}
% \begin{tabular}{|c|c|c|c|}
% 	\hline
% 	\textbf{Pessoa} & \textbf{Idade} & \textbf{Peso} & \textbf{Altura} \\ \hline
% 	Marcos & 26    & 68   & 178    \\ \hline
% 	Ivone  & 22    & 57   & 162    \\ \hline
% 	...    & ...   & ...  & ...    \\ \hline
% 	Sueli  & 40    & 65   & 153    \\ \hline
% \end{tabular}
% \fonte{Autor.}
% \end{quadro}

% Primeira opção, utilizando \texttt{autoref}: Ver o \autoref{quadro_exemplo}. 
% Segunda opção, utilizando  \texttt{ref}: Ver o Quadro \ref{quadro_exemplo}.
