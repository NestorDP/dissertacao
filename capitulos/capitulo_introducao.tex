\chapter{Introdução}\label{cap:intro}

% Apresentação do Tema e Contexto
% Nesta parte do documento, o pesquisador precisa explicar no texto a respeito do contexto 
% em que o tema está inserido. É bom imaginar que o leitor não conhece da área então 
% antes de explicar o assunto específico que será abordado, você deve explicar onde ele 
% está inserido.

Nos últimos anos novas técnicas para construção de robôs tem estado em muita evidência, em especial áreas como robótica móvel e robótica colaborativa têm chamado bastante atenção dos pesquisadores. Uma das principais características dessas áreas é a exigência de um alto grau de percepção do ambiente que rodeia o robô, além de uma execução mais precisa em seus movimentos, isto devido ao fato de que as atividades desempenhadas por estes robôs estão exigindo um nível cada vez maior de interação com as atividades desempenhadas por seres humanos.

Este grau de precisão requerido no desenvolvimento de novos robôs demandam um poder de processamento cada vez maior, consequentemente, elevando o consumo energético, o que pode vir a ser um problema principalmente em sistemas que fazem uso de baterias. Uma excelente alternativa para adicionar poder de processamento aliado ao baixo consumo é o uso do FPGA\@. O potencial que os FPGAs possuem para melhorar o desempenho de sistemas computacionais já é bem conhecido há algum tempo, as possibilidades de paralelismo e criação de estruturas de DSP dedicadas, são recursos muito interessantes que o hardware configurado oferece. 

Entretanto as facilidades de desenvolvimento encontradas em aplicações que fazem uso de softwares não estão disponíveis na mesma proporção no mundo do hardware configurável. A maio dificuldade no desenvolvimento de soluções que fazem uso do FPGA tornam os seus projetos mais longo, e necessitam de mão de obra extremamente especializada, o que fazem seu desenvolvimento mais caro. Isso faz com que o seu uso em projetos de robótica seja pouco usado e ate mesmo desistimulado. 

Atualmente o \textit{framework} ROS está se consolidando como o padrão na criação de novas plataformas robóticas, tanto no desenvolvimento de manipuladores colaborativos quanto na robótica móvel. O objetivo do ROS é facilitar a elaboração de novos robôs, através de um conjunto completo de ferramentas para desenvolvimento, como \textit{drivers} para sensores e atuadores, bibliotecas e principalmente reuso de código. Agrupar inúmeros ``blocos'' de softwares usados em robótica, fornecer drivers para hardwares específicos (sensores e atuadores), gerenciar troca de mensagens entre os nós que fazem parte do sistema, são as função do ROS\@. 

Estas características fazem com que o ROS seja reconhecido com um pseudo sistema operacional~\cite{rosPYO}. Dessa maneira o ROS se tornou muito ágil no desenvolvimento de novas aplicações para robótica. Usando aplicações já desenvolvidas e testadas por outros desenvolvedores, podemos criar novos sistemas completos apenas gerenciando estas aplicações na estrutura interna do ROS. Essa abordagem fez com que o número de pacotes para o ROS cresça em uma taxa muito rápida, desde o ano de seu lançamento, em 2007, até 2012 o ROS aumentou de 1 para 3699 pacotes~\cite{fpgarobotics}.

Aproveitar as facilidades de desenvolvimento proporcionadas pelo ROS em conjunto com o alto poder de processamento e baixo consumo que o FPGA oferece, seria um cenário ideal no desenvolvimento de novas aplicações com robôs. Para isso, precisamos estabelecer uma conexão com uma taxa de transferência de dados alta o suficiente para não influenciar de forma negativa no tempo de processamento e, que torne relativamente fácil seu uso por desenvolvedores especializados em robótica, mas sem grande experiência em FPGA. Portanto, este trabalho tem como objetivo estabelecer uma comunicação de alto desempenho entre ROS e um FPGA para que se possa aproveitar o melhor das duas tecnologias em projetos de robótica. 

explicar a parte da comunicação entre o computador/ros e o SoC melhorar a comunicação, comunicação eficiente, pacote pronto e de fácil integração com qualquer sistema ros, O mais genérico possível para se enquadrar a qualquer projeto é que o desenvolvedor tenha interesse em incluir um FPGA ao sistema



\begin{itemize}
    \item \textbf{Como estabelecer a comunicação entre o ROS e um sistema de processamento auxiliar 
embarcado em um FPGA?}

\end{itemize}

Este problema é o que o trabalho vai tentar resolver, possibilitando assim, o uso de aceleração por hardware através do fpga, ser incluída no desenvolvimento de novos projetos de robótica. Projetistas especializados em robótica poderão aproveitar dos benefícios do uso do hardware dedicado em seus projetos e profissionais que trabalham com descrição de hardware poderão desenvolver novas soluções para problemas de robótica de forma modularizada. 





\section{Justificativa}

Sistemas robóticos cada vez mais complexos exigem a necessidade do uso de processadores igualmente mais poderosos, consequentemente demandando um maior consumo de energia. Este aumento de consumo, provoca uma verdadeira briga entre poder de processamento e baixo consumo.

O FPGA é uma excelente alternativa, que pode oferecer aumento do poder de processamento em conjunto com baixo consumo de energia. \citeonline{dspFPGA} descreve algumas vantagens dos FPGAs modernos para uso em processamento digitais de sinais, como as cadeias de fast-carry usadas para implementar MACs de alta velocidade e o paralelismo tipicamente encontrado em dedign implementados em FPGA\@. Por essas características o FPGA necessita de frequências menores de trabalho para alcançar desempenho equivalente ou superior às soluções baseadas em processadores, diminuido a dissipação térmica, e consequentemente, necessitando um consumo de energia consideravelmente menor.

O tempo de desenvolvimento de projetos em FPGA é maior em relação a projetos puramente de software, por isso, a pesquisa busca ao final do projeto produzir um sistema genérico que possa ser usado em outras aplicações com poucas ou até mesmo nenhuma alteração se tornando uma alternativa para integrar o ROS a um FPGA, de forma simples e de baixo custo, possibilitando outras aplicações desta solução

Foi encontrado até o memento uma única pesquisa que relaciona o ROS e FPGA para  processamento de vídeo, nesse ponto a proposta deste trabalho difere da solução encontrado. No trabalho de \citeonline{fpgarobotics} são demonstradas três técnica para realizar a conexão entre o FPGA e o ROS, que se diferem da proposta por essa pesquisa. 


\section{Objetivos}

\subsection{Objetivo Geral}

Desenvolver uma solução para estabelecer comunicação entre \textit{Field Programmable Gate Array - FPGA}, 
configurado como um co-processador de vídeo.

\subsection{Objetivos Específicos}

\begin{itemize}
    \item Estudar teoria dos assuntos relevantes ao projeto: Verilog HDL, embedded linx,  Cyclone V, 
    TCP/IP Stack, ROS\@;
    \item Estudar conceito de programação de redes usando sockets em liguagem C++ e detalhes dos protocolos da rede TCP/IP usada para comunicação interna dos nós e serviços ROS\@;
    \item Implementar distribuição embedded linux para processador ARM embarcado no SoC Cyclone V da Intel;
    \item Estabelecer comunicação entre o ROS e o Cyclone V, através da tecnologia Gigabit Ethernet;
    \item Desenvolver aplicação em Verilog para testar comunicação;
    \item Avaliar a performance da rede entre o computador e o protótipo após a inclusão do FPGA ao sistema.
\end{itemize}


\section{Organização}

 No primeiro Capítulo~\ref{cap:intro}