\chapter{Introdução}



\section{Problema}

Agrupar inúmeros “blocos"de softwares usados em robótica, fornecer drivers para
hardwares específicos (sensores e atuadores), gerenciar troca de mensagens entre os nós
que fazem parte do sistema, são as função do ROS. Essas características fezem com que o ROS
seja reconhecido com um pseudo sistema operacional (PYO et al., 2017). Dessa maneira o
ROS se tornou muito ágil no desenvolvimento de novas aplicações para robótica. Usando
nós já desenvolvidos e testados por outros desenvolvedores podemos criar novos sistemas
completos apenas gerenciando esses nós na rede interna do ROS. Essa abordagem fez com
que o número de pacotes para o ROS cresça em uma taxa muito rápida, desde o ano de
seu lançamento, 2007, até 2012 o ROS aumentou de 1 para 3699 pacotes (YAMASHINA
et al., 2005).

Com essa distribuição de tarefas através de vários nós podemos criar sistemas cada
vez mais complexos, apenas inserindo novos nós na rede ROS, essa rede é gerenciada
pelo ROS Master, que é apensas mais um nó do sistema, mas com a função de ser um
servidor de nome e serviços para o restante dos nós. Ele identifica os nós na rede, assim
todos os nós podem se comunicar com os outros através de conexões peer-to-peer, Figura
1. Para desenvolver novas aplicações para integrarem o crescente grupo de pacotes ROS, o
desenvolvedor deve respeitar os protocolos de comunicação da rede, as bibliotecas do ROS
facilitam a vida do desenvolvedor, por já fornecer funções prontas para o desenvolvimento
de novos códigos compatíveis e que possam se registrar na rede. Detalhes dos protocolos e
interno podem ser visto em (ROS, 2011a), (ROS, 2018) e (ROS, 2011b).

Por se tratar de um hardware configurável o FPGA é ideal para processamento
digitais de sinais, e segundo Meyer-Baese (2007) "os Field programmable gate array -
FPGAs estão próximos a revolucionar o processamento digital de sinais, assim como os
DSPs fizeram algumas décadas atrás". O potencial que os FPGAs possuem para melhorar o 
desempenho de sistemas que utilizam processamento digitais de sinal é conhecido já algum
tempo, as possibilidades de paralelismo, criação de estruturas de DSP dedicadas à aplicação
são recursos muito interessantes que a possibilidade do hardware configurado oferecem,
mas em contra partida as facilidade de desenvolvimento encontradas em aplicações que
fazem uso de softwares não são encontradas nas mesmas proporções no mundo do hardware,
sendo assim:

\begin{itemize}
    \item \textbf{Como estabelecer a comunicação entre o ROS e um sistema de processamento de vídeo 
embarcado em um FPGA?}


Esse problema inicial nos leva naturalmente a outro:


    \item \textbf{Stabelecendo a comunicação entre o FPGA e o ROS a execução do pro-
cessamento de vídeo de forma paralela, embarcada em um FPGA, pode
melhorar a performance do sistema?}
  
\end{itemize}







\section{Justificativa}

Nos últimos anos novas técnicas para construção de robôs tem sido bastante
estudadas (YAMASHINA et al., 2005), em especial uma área da robótica que tem sido
bastante explorada é a robótica móvel. A principal características que tem sido buscada é
cada vez fornecer mais autonomia ao sistemas robóticos o que vem tornado cada vez seus
softwares mais complexos, o que aumenta a necessidade do uso de processadores cada vez
mais poderosos, que consomem mais energia. Entretanto a busca por mais autonomia, diz
respeito também às baterias, que são as fontes de energia da maioria dos robôs móveis, o
que provoca uma verdadeira briga entre poder de processamento e baixo consumo.

Sendo assim, o FPGA pode ser uma ótima alternativa para solucionar os problemas
de aumento do poder de processamento em conjunto com baixo consumo de energia.
Meyer-Baese (2007) descreve algumas vantagens dos FPGAs modernos para uso em
processamento digitais de sinais, como as cadeias de fast-carry usadas para implementar
MACs de alta velocidade e o paralelismo tipicamente encontrado em dedign implementados
em FPGA. Por essas características o FPGA necessita de frequências menores de trabalho
para alcançar desempenho equivalente ou superior às soluções baseadas em processadores,
tornando a dissipação de energia menor.

Foi encontrado até o memento uma única pesquisa que relaciona o ROS e FPGA
para processamento de vídeo, nesse ponto a proposta deste trabalho difere da solução
encontrado. No trabalho de Yamashina et al. (2005), são demonstradas três técnica para
realizar a conexão entre o FPGA e o ROS, que se diferem da proposta por essa pesquisa. A
ideia é utilizar um soft-core processor, que é um processador descrito em linguagem HDL
embarcado no FPGA (CHU, 2012), esse processador ficará responsável por estabelecer a
comunicação entre a rede TCP/IP e o hardware configurado no FPGA.

O FPGA que será utilizado para o desenvolvimento da pesquisa é o Cyclone IV
da Intel. A Intel fornece o Nios II um soft-core porocessor para ser utilizado em conjunto
com os seus FPGAs. O Nios II é disponibilizado em duas versões, a fast: que foi projetado
para alta performance e a economy: projetado para ocupar um menor espaço dentro do
FPGA(CHU, 2012).

A versão do Nios II econômica foi escolhida por não necessitar de uma licença
adicional para seu uso, uma alternativa para RTOS é o FreeRTOS, que é um sistema
operacional de tempo real de código aberto(BARRY, 2016b) e para stack TCP/IP a
Lightweight TCP/IP stack - LwIP, que é uma versão do pacote de protocolos TCP/IP
de código aberto para ser usado em sistemas embarcado(NONGNU, 2018). Todas as
ferramentas de softwares necessárias para desenvolver o projeto são de uso livre, o objetivo 
é fazer uso de o maior número de ferramentas sem custos adicionais, como licenças e
softwares pagos.

O tempo de desenvolvimento de projetos em FPGA é maior em relação a projetos
puramente de software, por isso, a pesquisa busca ao final do projeto produzir um sistema
genérico que possa ser usado em outras aplicações com poucas ou até mesmo nenhuma
alteração se tornando uma alternativa para integrar o ROS a um FPGA, de forma simples
e de baixo custo, possibilitando outras aplicações desta solução.

\section{Objetivos}

\subsection{Objetivo Geral}

Desenvolver uma solução para estabelecer comunicação entre \textit{Field Programmable
Gate Array - FPGA}, configurado como um co-processador de vídeo e o  \textit{Robot Operating
System - ROS} avaliando o impacto desta aplicação ao sistema.

\subsection{Objetivos Específicos}

\begin{itemize}
    \item Estudar os assuntos relevantes ao projeto: Verilog HDL, RTOS, Nios II, TCO/IP Stack, ROS;
    \item Conhecer com detalhes os protocolos da rede TCP/IP usada para comunicação interna dos nós e serviços ROS;
    \item Desenvolver plataforma com Nios II como base para o andamento do projeto;
    \item Implementa um sistema operacional de tempo real - RTOS na plataforma base;
    \item Estabelecer comunicação entre o ROS e o sistema Nios II (embarcador no FPGA) através da tecnologia Gigabit Ethernet;
    \item Testar aplicações de processamento de vídeo em hardware em conjunto com ROS;
    \item Avaliar a performance com a inclusão do FPGA ao sistema.
\end{itemize}


\section{Organização}