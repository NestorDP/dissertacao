\chapter{Robot Operating System - ROS}\label{cap:ros}



\section{Sistema multiagentes}


% a busca por sistemas com um grau maior de autonomiaA robótica tem se caracterizádo pelo grande nível de percepção do ambiente e pelos sistemacomplexos de controle do movimentos 


% Com essa distribuição de tarefas através de vários nós podemos criar sistemas cada
% vez mais complexos, apenas inserindo novos nós na rede, essa rede é gerenciada pelo ROS Master, que é apensas mais um nó do sistema, mas com a função de ser um servidor de nome e serviços para o restante dos nós. Ele identifica os nós na rede, assim todos os nós podem se comunicar com os outros através de conexões peer-to-peer, igura
% 1. Para desenvolver novas aplicações para o crescente grupo de pacotes ROS, o desenvolvedor deve respeitar os protocolos de comunicação da rede, as bibliotecas do ROS facilitam este trabalho, por já fornecer funções prontas para o desenvolvimento
% de novos códigos compatíveis e que possam se registrar na rede. Detalhes dos  rotocolos e interno podem ser visto em (ROS, 2011a), (ROS, 2018) e (ROS, 2011b).