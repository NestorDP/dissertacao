\chapter{Arquitetura do sistema}\label{cap:arquitetura}

Para conseguirmos estabelecer a comunicação entre o computador e o SoC precisaremos efetuar programação de sockets e bibliotecas específicas para trabalho em redes, desenvolver um pacote ROS para disponibilizar os dados recebidos através da interface de rede para os outros pacotes ROS do sistema robótico, além de um programa rodando no HPS do SoC para estabelecer esta comunicação entre a interface de rede da placa De10-nano e a aplicação sendo executada no FPGA\@. Já a aplicação que estará embarcada no FPGA contido no SoC deverá ser descrita por alguma linguagem de descrição de hardware, como por exemplo, verilog ou VHDL\@.

Todas essas etapas descritas anteriormente são necessárias para a construção completa do sistema proposto, o que torna o desenvolvimento da solução completa um desafio devido às diferentes ferramentas de software e hardware necessárias para sua conclusão. Tendo em vista este problema, a solução foi idealizada para conter o maior grau de modularidade possível, ou seja, cada uma dessas etapas será tratada com um projeto independente, apenas tendo cuidado para garantir a correta comunicação entre cada uma delas.

A grande vantagem que esse abordagem traz ao projeto é a possibilidade futura, de tanto a continuação do desenvolvimento como da manutenção do sistema, serem realizados por profissionais com background nas diferentes áreas envolvidas, sem a necessidade de se envolver no desenvolvimento de outros módulos. Sendo assim, um profissional especialista em descrição de hardware poderia se dedicar apenas à concepção da solução embarcada no FPGA, sem a necessidade possuir conhecimento em programação de redes. 

\section{Modelo cliente-servidor}
A comunicação entre o host, rodando o ROS, e a placa DE10-nano será estabelecida através de uma rede gigabit ethernet ponto a ponto, ou seja, o host e o SoC estarão conectados diretamente entre si. Desta maneira é possível obter o melhor desempenho da rede, alcançando as maiores taxas de transmissão de dados. Com o meio de comunicação definido é preciso definir também a arquitetura da comunicação, uma boa alternativa é o modelo cliente-servidor.

O modelo cliente-servidor é caracterizado por possuir uma estrutura que permite dividir o trabalho computacional entre os participantes da comunicação, isto é, entre o servidor, que é o encarregado de disponibilizar os recursos e serviços, e o cliente, que realiza as solicitações para os serviços disponíveis.  Desta maneira tanto o cliente quanto o servidor foram tratados como módulos independentes durante o desenvolvimento do trabalho. O uso do modelo cliente-servidor contribui de forma significativa para que o sistema alcance o máximo de modularização, essa abordagem facilita, entre outras coisas, a depuração e manutenção do código, o que proporciona mais agilidade e simplicidade no processo de desenvolvimento da solução.


Na Figura~\ref{fig:arquitetura} podemos ter uma visão global do sistema, nela podemos ver cada etapa da comunicação.No lado do host, está instalado o ROS, nele também é onde o cliente será executado, assim sendo o cliente fica responsável por ler o tópico de entrada, fornecido por outro nó do sistema, realizar uma solicitação ao servidor enviando os dados já lidos. O servidor, por sua vez, aceita a solicitação do cliente, recebe os dados e os envia à aplicação embarcada no FPGA que os devolve após seu processamento. Para completar o ciclo o servidor retorna os dados processados ao cliente, que por sua vez, disponibiliza os dados já processados através do tópico de saída. 

\begin{figure}[ht]
	\caption{Arquitetura geral}
	\begin{center}
		\includegraphics[scale=0.7]{imagens/arquitetura_geral.png}\\
		{\small \textbf{Fonte:} do autor}
    \end{center}\label{fig:arquitetura}
\end{figure}




\section{Biblioteca de comunicação - libinterfacesocket}

Para manter o padrão do desenvolvimentos dos códigos tanto do cliente quanto do servidor, foi desenvolvida uma classe, que fornece os métodos para a abertura da comunicação, além de métodos para envio e recebimento das mensagens através da rede gigabit ethernet. Essa classe foi desenvolvida como um módulo a parte e compilada como uma biblioteca estática, sendo assim, a partir do momento em que os métodos de comunicação estiverem testados e validado tanto o código do cliente quanto o do servidor poderão fazer uso desta biblioteca, eliminando assim a necessidade de reescrever uma parte do código código. Outra vantagem nessa abordagem é que ao manter o código desassociado tanto do servidor como do cliente, nós possibilita fazer alterações ou correções de bugs, sem necessariamente realizar alterações nos códigos do servidor ou do cliente.

A programação da biblioteca foi realizada com base em sockets. Sockets são um caminho para conectar processos em uma rede de computadores. A conexão através de sockets entre nós em uma rede independe do protocolo. Um nó da rede ouve uma determinada porta para um IP específico esperando por o pedido de conexão do segundo nó, assim a conexão entre dois processos é estabelecida. O servidor é o nó que aguarda o pedido ser enviado pelo cliente.

A programação de sockets em C++ possibilita um alto nível de otimização da comunicação entre os processos, principalmente por se tratar de um modelo cliente-servidor onde só existirá a comunicação entre o servidor e apenas um cliente. Após implementar a comunicação entre o servidor e o cliente, poderá ser testadas novas técnicas de para otimizar o desempenho da rede possibilitando o aumento da taxa de transferência de dados entre o servidor e o cliente.

O código fonte da biblioteca pode ser encontrado no repositório no github~\cite{Pereira-Neto-Biblioteca}, que pode ser visto na figura~\ref{fig:gitlib}, onde podemos observar a estrutura de arquivos da blibioteca. Vale frizar que, a libinterfacesocket possui um makefile para realizar o processo de compilação de forma automática. Assim podemos de forma simplificada compilar e instalar a biblioteca tanto no sistema do host onde será executado o cliente, quanto no sistema do HPS embarcado no SoC, onde o servidor estará rodando. 

\begin{figure}[ht]
	\caption{Repositório libinterfacesocket}
	\begin{center}
		\includegraphics[scale=0.26]{imagens/git_libinterfacesocket.png}\\
		{\small \textbf{Fonte:} do autor}
    \end{center}\label{fig:gitlib}
\end{figure}
