\chapter{Introdução}

\section{Estado da Arte}

\section{Problema}

\section{Justificativa}

\section{Objetivos}

\subsection{Objetivo Geral}

Desenvolver uma solução para estabelecer comunicação entre \textit{Field Programmable
Gate Array - FPGA}, configurado como um co-processador de vídeo e o  \textit{Robot Operating
System - ROS} avaliando o impacto desta aplicação ao sistema.

\subsection{Objetivos Específicos}

\begin{itemize}
    \item Estudar os assuntos relevantes ao projeto: Verilog HDL, RTOS, Nios II, TCO/IP Stack, ROS;
    \item Conhecer com detalhes os protocolos da rede TCP/IP usada para comunicação interna dos nós e serviços ROS;
    \item Desenvolver plataforma com Nios II como base para o andamento do projeto;
    \item Implementa um sistema operacional de tempo real - RTOS na plataforma base;
    \item Estabelecer comunicação entre o ROS e o sistema Nios II (embarcador no FPGA) através da tecnologia Gigabit Ethernet;
    \item Testar aplicações de processamento de vídeo em hardware em conjunto com ROS;
    \item Avaliar a performance com a inclusão do FPGA ao sistema.
\end{itemize}


\section{Organização}



